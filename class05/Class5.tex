% Options for packages loaded elsewhere
\PassOptionsToPackage{unicode}{hyperref}
\PassOptionsToPackage{hyphens}{url}
\PassOptionsToPackage{dvipsnames,svgnames,x11names}{xcolor}
%
\documentclass[
  letterpaper,
  DIV=11,
  numbers=noendperiod]{scrartcl}

\usepackage{amsmath,amssymb}
\usepackage{lmodern}
\usepackage{iftex}
\ifPDFTeX
  \usepackage[T1]{fontenc}
  \usepackage[utf8]{inputenc}
  \usepackage{textcomp} % provide euro and other symbols
\else % if luatex or xetex
  \usepackage{unicode-math}
  \defaultfontfeatures{Scale=MatchLowercase}
  \defaultfontfeatures[\rmfamily]{Ligatures=TeX,Scale=1}
\fi
% Use upquote if available, for straight quotes in verbatim environments
\IfFileExists{upquote.sty}{\usepackage{upquote}}{}
\IfFileExists{microtype.sty}{% use microtype if available
  \usepackage[]{microtype}
  \UseMicrotypeSet[protrusion]{basicmath} % disable protrusion for tt fonts
}{}
\makeatletter
\@ifundefined{KOMAClassName}{% if non-KOMA class
  \IfFileExists{parskip.sty}{%
    \usepackage{parskip}
  }{% else
    \setlength{\parindent}{0pt}
    \setlength{\parskip}{6pt plus 2pt minus 1pt}}
}{% if KOMA class
  \KOMAoptions{parskip=half}}
\makeatother
\usepackage{xcolor}
\setlength{\emergencystretch}{3em} % prevent overfull lines
\setcounter{secnumdepth}{-\maxdimen} % remove section numbering
% Make \paragraph and \subparagraph free-standing
\ifx\paragraph\undefined\else
  \let\oldparagraph\paragraph
  \renewcommand{\paragraph}[1]{\oldparagraph{#1}\mbox{}}
\fi
\ifx\subparagraph\undefined\else
  \let\oldsubparagraph\subparagraph
  \renewcommand{\subparagraph}[1]{\oldsubparagraph{#1}\mbox{}}
\fi

\usepackage{color}
\usepackage{fancyvrb}
\newcommand{\VerbBar}{|}
\newcommand{\VERB}{\Verb[commandchars=\\\{\}]}
\DefineVerbatimEnvironment{Highlighting}{Verbatim}{commandchars=\\\{\}}
% Add ',fontsize=\small' for more characters per line
\usepackage{framed}
\definecolor{shadecolor}{RGB}{241,243,245}
\newenvironment{Shaded}{\begin{snugshade}}{\end{snugshade}}
\newcommand{\AlertTok}[1]{\textcolor[rgb]{0.68,0.00,0.00}{#1}}
\newcommand{\AnnotationTok}[1]{\textcolor[rgb]{0.37,0.37,0.37}{#1}}
\newcommand{\AttributeTok}[1]{\textcolor[rgb]{0.40,0.45,0.13}{#1}}
\newcommand{\BaseNTok}[1]{\textcolor[rgb]{0.68,0.00,0.00}{#1}}
\newcommand{\BuiltInTok}[1]{\textcolor[rgb]{0.00,0.23,0.31}{#1}}
\newcommand{\CharTok}[1]{\textcolor[rgb]{0.13,0.47,0.30}{#1}}
\newcommand{\CommentTok}[1]{\textcolor[rgb]{0.37,0.37,0.37}{#1}}
\newcommand{\CommentVarTok}[1]{\textcolor[rgb]{0.37,0.37,0.37}{\textit{#1}}}
\newcommand{\ConstantTok}[1]{\textcolor[rgb]{0.56,0.35,0.01}{#1}}
\newcommand{\ControlFlowTok}[1]{\textcolor[rgb]{0.00,0.23,0.31}{#1}}
\newcommand{\DataTypeTok}[1]{\textcolor[rgb]{0.68,0.00,0.00}{#1}}
\newcommand{\DecValTok}[1]{\textcolor[rgb]{0.68,0.00,0.00}{#1}}
\newcommand{\DocumentationTok}[1]{\textcolor[rgb]{0.37,0.37,0.37}{\textit{#1}}}
\newcommand{\ErrorTok}[1]{\textcolor[rgb]{0.68,0.00,0.00}{#1}}
\newcommand{\ExtensionTok}[1]{\textcolor[rgb]{0.00,0.23,0.31}{#1}}
\newcommand{\FloatTok}[1]{\textcolor[rgb]{0.68,0.00,0.00}{#1}}
\newcommand{\FunctionTok}[1]{\textcolor[rgb]{0.28,0.35,0.67}{#1}}
\newcommand{\ImportTok}[1]{\textcolor[rgb]{0.00,0.46,0.62}{#1}}
\newcommand{\InformationTok}[1]{\textcolor[rgb]{0.37,0.37,0.37}{#1}}
\newcommand{\KeywordTok}[1]{\textcolor[rgb]{0.00,0.23,0.31}{#1}}
\newcommand{\NormalTok}[1]{\textcolor[rgb]{0.00,0.23,0.31}{#1}}
\newcommand{\OperatorTok}[1]{\textcolor[rgb]{0.37,0.37,0.37}{#1}}
\newcommand{\OtherTok}[1]{\textcolor[rgb]{0.00,0.23,0.31}{#1}}
\newcommand{\PreprocessorTok}[1]{\textcolor[rgb]{0.68,0.00,0.00}{#1}}
\newcommand{\RegionMarkerTok}[1]{\textcolor[rgb]{0.00,0.23,0.31}{#1}}
\newcommand{\SpecialCharTok}[1]{\textcolor[rgb]{0.37,0.37,0.37}{#1}}
\newcommand{\SpecialStringTok}[1]{\textcolor[rgb]{0.13,0.47,0.30}{#1}}
\newcommand{\StringTok}[1]{\textcolor[rgb]{0.13,0.47,0.30}{#1}}
\newcommand{\VariableTok}[1]{\textcolor[rgb]{0.07,0.07,0.07}{#1}}
\newcommand{\VerbatimStringTok}[1]{\textcolor[rgb]{0.13,0.47,0.30}{#1}}
\newcommand{\WarningTok}[1]{\textcolor[rgb]{0.37,0.37,0.37}{\textit{#1}}}

\providecommand{\tightlist}{%
  \setlength{\itemsep}{0pt}\setlength{\parskip}{0pt}}\usepackage{longtable,booktabs,array}
\usepackage{calc} % for calculating minipage widths
% Correct order of tables after \paragraph or \subparagraph
\usepackage{etoolbox}
\makeatletter
\patchcmd\longtable{\par}{\if@noskipsec\mbox{}\fi\par}{}{}
\makeatother
% Allow footnotes in longtable head/foot
\IfFileExists{footnotehyper.sty}{\usepackage{footnotehyper}}{\usepackage{footnote}}
\makesavenoteenv{longtable}
\usepackage{graphicx}
\makeatletter
\def\maxwidth{\ifdim\Gin@nat@width>\linewidth\linewidth\else\Gin@nat@width\fi}
\def\maxheight{\ifdim\Gin@nat@height>\textheight\textheight\else\Gin@nat@height\fi}
\makeatother
% Scale images if necessary, so that they will not overflow the page
% margins by default, and it is still possible to overwrite the defaults
% using explicit options in \includegraphics[width, height, ...]{}
\setkeys{Gin}{width=\maxwidth,height=\maxheight,keepaspectratio}
% Set default figure placement to htbp
\makeatletter
\def\fps@figure{htbp}
\makeatother

\KOMAoption{captions}{tableheading}
\makeatletter
\makeatother
\makeatletter
\makeatother
\makeatletter
\@ifpackageloaded{caption}{}{\usepackage{caption}}
\AtBeginDocument{%
\ifdefined\contentsname
  \renewcommand*\contentsname{Table of contents}
\else
  \newcommand\contentsname{Table of contents}
\fi
\ifdefined\listfigurename
  \renewcommand*\listfigurename{List of Figures}
\else
  \newcommand\listfigurename{List of Figures}
\fi
\ifdefined\listtablename
  \renewcommand*\listtablename{List of Tables}
\else
  \newcommand\listtablename{List of Tables}
\fi
\ifdefined\figurename
  \renewcommand*\figurename{Figure}
\else
  \newcommand\figurename{Figure}
\fi
\ifdefined\tablename
  \renewcommand*\tablename{Table}
\else
  \newcommand\tablename{Table}
\fi
}
\@ifpackageloaded{float}{}{\usepackage{float}}
\floatstyle{ruled}
\@ifundefined{c@chapter}{\newfloat{codelisting}{h}{lop}}{\newfloat{codelisting}{h}{lop}[chapter]}
\floatname{codelisting}{Listing}
\newcommand*\listoflistings{\listof{codelisting}{List of Listings}}
\makeatother
\makeatletter
\@ifpackageloaded{caption}{}{\usepackage{caption}}
\@ifpackageloaded{subcaption}{}{\usepackage{subcaption}}
\makeatother
\makeatletter
\@ifpackageloaded{tcolorbox}{}{\usepackage[many]{tcolorbox}}
\makeatother
\makeatletter
\@ifundefined{shadecolor}{\definecolor{shadecolor}{rgb}{.97, .97, .97}}
\makeatother
\makeatletter
\makeatother
\ifLuaTeX
  \usepackage{selnolig}  % disable illegal ligatures
\fi
\IfFileExists{bookmark.sty}{\usepackage{bookmark}}{\usepackage{hyperref}}
\IfFileExists{xurl.sty}{\usepackage{xurl}}{} % add URL line breaks if available
\urlstyle{same} % disable monospaced font for URLs
\hypersetup{
  pdftitle={Class 5: Data Visualization},
  pdfauthor={Sejal (A15945666)},
  colorlinks=true,
  linkcolor={blue},
  filecolor={Maroon},
  citecolor={Blue},
  urlcolor={Blue},
  pdfcreator={LaTeX via pandoc}}

\title{Class 5: Data Visualization}
\author{Sejal (A15945666)}
\date{}

\begin{document}
\maketitle
\ifdefined\Shaded\renewenvironment{Shaded}{\begin{tcolorbox}[frame hidden, borderline west={3pt}{0pt}{shadecolor}, sharp corners, boxrule=0pt, enhanced, breakable, interior hidden]}{\end{tcolorbox}}\fi

\hypertarget{base-r-graphics-vs-ggplot2}{%
\subsection{Base R graphics vs
ggplot2}\label{base-r-graphics-vs-ggplot2}}

There are many graphics systems available in R, including so-called
``base'' R graphics and the very popular \textbf{ggplot2} package.

To compare these let's play with the inbuilt\texttt{cars} dataset.

\begin{Shaded}
\begin{Highlighting}[]
\FunctionTok{head}\NormalTok{(cars)}
\end{Highlighting}
\end{Shaded}

\begin{verbatim}
  speed dist
1     4    2
2     4   10
3     7    4
4     7   22
5     8   16
6     9   10
\end{verbatim}

To use ``base'' R I can simply call the \texttt{plot()} function''

\begin{Shaded}
\begin{Highlighting}[]
\FunctionTok{plot}\NormalTok{(cars)}
\end{Highlighting}
\end{Shaded}

\begin{figure}[H]

{\centering \includegraphics{Class5_files/figure-pdf/unnamed-chunk-2-1.pdf}

}

\end{figure}

To use \texttt{ggplot} package I first need to install it with the
function \texttt{install.packages("ggplot2")}.

I will run this in the R console (i.e.~the R brain) as I do not want to
re-install it every time I render my report\ldots{}

The main function in this package is called \texttt{ggplot()}. Can I
just call it

\begin{Shaded}
\begin{Highlighting}[]
\FunctionTok{library}\NormalTok{(ggplot2)}
\FunctionTok{ggplot}\NormalTok{()}
\end{Highlighting}
\end{Shaded}

\begin{figure}[H]

{\centering \includegraphics{Class5_files/figure-pdf/unnamed-chunk-3-1.pdf}

}

\end{figure}

To make a figure with ggplot I need always at least 3 things.

\begin{itemize}
\tightlist
\item
  \textbf{data} (i.e.~what I want to plot)
\item
  \textbf{aes} the aesthetic mapping of the data to the plot I want.
\item
  \textbf{the geoms} i.e How I want to plot the data
\end{itemize}

\begin{Shaded}
\begin{Highlighting}[]
\FunctionTok{ggplot}\NormalTok{(}\AttributeTok{data=}\NormalTok{cars)}\SpecialCharTok{+}
  \FunctionTok{aes}\NormalTok{(}\AttributeTok{x=}\NormalTok{speed, }\AttributeTok{y=}\NormalTok{dist)}\SpecialCharTok{+}
  \FunctionTok{geom\_point}\NormalTok{()}
\end{Highlighting}
\end{Shaded}

\begin{figure}[H]

{\centering \includegraphics{Class5_files/figure-pdf/unnamed-chunk-4-1.pdf}

}

\end{figure}

If I want to more things I can keep adding layers, e.g.~

\begin{Shaded}
\begin{Highlighting}[]
\FunctionTok{ggplot}\NormalTok{(}\AttributeTok{data=}\NormalTok{cars)}\SpecialCharTok{+}
  \FunctionTok{aes}\NormalTok{(}\AttributeTok{x=}\NormalTok{speed, }\AttributeTok{y=}\NormalTok{dist)}\SpecialCharTok{+}
  \FunctionTok{geom\_point}\NormalTok{()}\SpecialCharTok{+}
  \FunctionTok{geom\_smooth}\NormalTok{()}
\end{Highlighting}
\end{Shaded}

\begin{verbatim}
`geom_smooth()` using method = 'loess' and formula = 'y ~ x'
\end{verbatim}

\begin{figure}[H]

{\centering \includegraphics{Class5_files/figure-pdf/unnamed-chunk-5-1.pdf}

}

\end{figure}

GG plot is much more verbose than base R plots for standard plots but it
has a consistent layer system that I can use to make just about any
plot.

Let's make a pliot with a straight line fit - i.e a linear model and no
standard error shown.

\begin{Shaded}
\begin{Highlighting}[]
\FunctionTok{ggplot}\NormalTok{(}\AttributeTok{data=}\NormalTok{cars)}\SpecialCharTok{+}
  \FunctionTok{aes}\NormalTok{(}\AttributeTok{x=}\NormalTok{speed, }\AttributeTok{y=}\NormalTok{dist)}\SpecialCharTok{+}
  \FunctionTok{geom\_point}\NormalTok{()}\SpecialCharTok{+}
  \FunctionTok{geom\_smooth}\NormalTok{(}\AttributeTok{method=}\StringTok{"lm"}\NormalTok{, }\AttributeTok{se=}\ConstantTok{FALSE}\NormalTok{)}\SpecialCharTok{+}
  \FunctionTok{labs}\NormalTok{(}\AttributeTok{title=}\StringTok{"Distance versus Speed for Older Cars"}\NormalTok{, }\AttributeTok{subtitle=}\StringTok{"Best Fit Model"}\NormalTok{, }\AttributeTok{x=} \StringTok{"Speed mph"}\NormalTok{, }\AttributeTok{y=}\StringTok{"Distance meters"}\NormalTok{)}\SpecialCharTok{+}
  \FunctionTok{theme\_bw}\NormalTok{()}
\end{Highlighting}
\end{Shaded}

\begin{verbatim}
`geom_smooth()` using formula = 'y ~ x'
\end{verbatim}

\begin{figure}[H]

{\centering \includegraphics{Class5_files/figure-pdf/unnamed-chunk-6-1.pdf}

}

\end{figure}

\hypertarget{a-more-complicated-plot}{%
\subsection{A more complicated plot}\label{a-more-complicated-plot}}

Let's plot some gene expression data The code below reads the results of
a differential expression analysis where a new anti-viral drug is being
tested.

\begin{Shaded}
\begin{Highlighting}[]
\NormalTok{url }\OtherTok{\textless{}{-}} \StringTok{"https://bioboot.github.io/bimm143\_S20/class{-}material/up\_down\_expression.txt"}
\NormalTok{genes }\OtherTok{\textless{}{-}} \FunctionTok{read.delim}\NormalTok{(url)}
\FunctionTok{head}\NormalTok{(genes)}
\end{Highlighting}
\end{Shaded}

\begin{verbatim}
        Gene Condition1 Condition2      State
1      A4GNT -3.6808610 -3.4401355 unchanging
2       AAAS  4.5479580  4.3864126 unchanging
3      AASDH  3.7190695  3.4787276 unchanging
4       AATF  5.0784720  5.0151916 unchanging
5       AATK  0.4711421  0.5598642 unchanging
6 AB015752.4 -3.6808610 -3.5921390 unchanging
\end{verbatim}

\begin{quote}
Q. how many genes are in the dataset?
\end{quote}

\begin{Shaded}
\begin{Highlighting}[]
\FunctionTok{nrow}\NormalTok{(genes)}
\end{Highlighting}
\end{Shaded}

\begin{verbatim}
[1] 5196
\end{verbatim}

\begin{quote}
Q How can we summerize that last column - the ``State'' column?
\end{quote}

\begin{Shaded}
\begin{Highlighting}[]
\FunctionTok{table}\NormalTok{(genes[,}\StringTok{"State"}\NormalTok{])}
\end{Highlighting}
\end{Shaded}

\begin{verbatim}

      down unchanging         up 
        72       4997        127 
\end{verbatim}

\begin{Shaded}
\begin{Highlighting}[]
\NormalTok{p }\OtherTok{\textless{}{-}} \FunctionTok{ggplot}\NormalTok{(genes)}\SpecialCharTok{+}
  \FunctionTok{aes}\NormalTok{(}\AttributeTok{x=}\NormalTok{Condition1, }\AttributeTok{y=}\NormalTok{Condition2, }\AttributeTok{col=}\NormalTok{State)}\SpecialCharTok{+}
  \FunctionTok{geom\_point}\NormalTok{()}
\end{Highlighting}
\end{Shaded}

I can now just call \texttt{p} when I want to plot or add to it.

\begin{Shaded}
\begin{Highlighting}[]
\NormalTok{p }\SpecialCharTok{+} \FunctionTok{labs}\NormalTok{(}\AttributeTok{title=}\StringTok{"Gene expression changes upon drug treatment"}\NormalTok{, }\AttributeTok{x=}\StringTok{"Control"}\NormalTok{, }\AttributeTok{y=}\StringTok{"Drug treated"}\NormalTok{)}\SpecialCharTok{+} \FunctionTok{scale\_colour\_manual}\NormalTok{( }\AttributeTok{values=}\FunctionTok{c}\NormalTok{(}\StringTok{"blue"}\NormalTok{,}\StringTok{"gray"}\NormalTok{,}\StringTok{"red"}\NormalTok{) )}
\end{Highlighting}
\end{Shaded}

\begin{figure}[H]

{\centering \includegraphics{Class5_files/figure-pdf/unnamed-chunk-11-1.pdf}

}

\end{figure}

\hypertarget{going-further}{%
\subsection{Going Further}\label{going-further}}

Here I read a slightly larger dataset

\begin{Shaded}
\begin{Highlighting}[]
\CommentTok{\# File location online}
\NormalTok{url }\OtherTok{\textless{}{-}} \StringTok{"https://raw.githubusercontent.com/jennybc/gapminder/master/inst/extdata/gapminder.tsv"}

\NormalTok{gapminder }\OtherTok{\textless{}{-}} \FunctionTok{read.delim}\NormalTok{(url)}
\FunctionTok{head}\NormalTok{(gapminder)}
\end{Highlighting}
\end{Shaded}

\begin{verbatim}
      country continent year lifeExp      pop gdpPercap
1 Afghanistan      Asia 1952  28.801  8425333  779.4453
2 Afghanistan      Asia 1957  30.332  9240934  820.8530
3 Afghanistan      Asia 1962  31.997 10267083  853.1007
4 Afghanistan      Asia 1967  34.020 11537966  836.1971
5 Afghanistan      Asia 1972  36.088 13079460  739.9811
6 Afghanistan      Asia 1977  38.438 14880372  786.1134
\end{verbatim}

\begin{Shaded}
\begin{Highlighting}[]
\FunctionTok{ggplot}\NormalTok{(gapminder)}\SpecialCharTok{+}
  \FunctionTok{aes}\NormalTok{(}\AttributeTok{x=}\NormalTok{gdpPercap, }\AttributeTok{y=}\NormalTok{lifeExp, }\AttributeTok{color=}\NormalTok{continent, }\AttributeTok{size=}\NormalTok{pop)}\SpecialCharTok{+}
  \FunctionTok{geom\_point}\NormalTok{(}\AttributeTok{alpha=}\FloatTok{0.5}\NormalTok{)}
\end{Highlighting}
\end{Shaded}

\begin{figure}[H]

{\centering \includegraphics{Class5_files/figure-pdf/unnamed-chunk-13-1.pdf}

}

\end{figure}

A very useful layer to add sometimes is for ``faceting''.

\begin{Shaded}
\begin{Highlighting}[]
\FunctionTok{ggplot}\NormalTok{(gapminder)}\SpecialCharTok{+}
  \FunctionTok{aes}\NormalTok{(}\AttributeTok{x=}\NormalTok{gdpPercap, }\AttributeTok{y=}\NormalTok{lifeExp, }\AttributeTok{color=}\NormalTok{continent, }\AttributeTok{size=}\NormalTok{pop)}\SpecialCharTok{+}
  \FunctionTok{geom\_point}\NormalTok{(}\AttributeTok{alpha=}\FloatTok{0.5}\NormalTok{)}\SpecialCharTok{+}
  \FunctionTok{facet\_wrap}\NormalTok{(}\SpecialCharTok{\textasciitilde{}}\NormalTok{continent)}
\end{Highlighting}
\end{Shaded}

\begin{figure}[H]

{\centering \includegraphics{Class5_files/figure-pdf/unnamed-chunk-14-1.pdf}

}

\end{figure}



\end{document}
